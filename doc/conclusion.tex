\chapter{Conclusion}

En conclusion de ce rapport, nous souhaitons faire un bilan des apports techniques, organisationnels et humains de ce projet ainsi que des difficultés rencontrées

D'un point de vu purement technique, ce projet nous a appris l'utilisation de Git et de \texttt{make} via le \texttt{Makefile}, des outils nouveaux pour deux étudiants tout droit sortis des classes préparatoires. La gestion des conflits a été particulièrement formateur, tout comme celle de la compilation séparée aidée par le \texttt{Makefile}. Pour ce qui est de la partie programmation en elle-même, c'est la mise en pratique des connaissances vues en cours et parfois abstraites qui ont été le plus enrichissant. Savoir quand et comment utiliser des structures, des pointeurs (variables et fonctions), utiliser des librairies (\texttt{getopt}, \texttt{SDL}) afin d'optimiser au mieux nos codes sont une première expérience qui nous semble fondamental pour la consolidation de nos connaissances. On a aussi pu sentir l'aspect \og orienté objet\fg{} que proposait ce projet, notamment avec la structure abstraite des règles permettant de les modifier isolément du reste du programme. Enfin, tester notre code est aussi une compétence importante que l'on a commencé à acquérir grâce à ce projet, même si ce point nous a posé des difficultés, notamment savoir exactement quoi tester dans un code, et encore plus comment le tester. 

Mais gérer ce premier projet nous a surtout obligé a acquérir des compétences organisationnelles et humaines essentielles pour le mener à bien. Respecter les contraintes, anticiper l'évolution du code et le tester au fil de son avancement ont été des compétences à apprendre sur le tas, surtout pour une première véritable expérience de projet. Penser à commenter de manière juste et concise nos algorithmes a été une autre de nos difficultés, même si peu contraignantes dans le cadre d'un projet comme celui-ci. La répartition de notre code dans de nombreux fichiers nous a posé quelques difficultés. Cela nous a poussé à repenser l'organisation de notre code et de réaliser l'importance d'un graphe de dépendances.

Pour finir, un projet en binôme se mène à deux, et il est fondamental de gérer le travail de chacun pour ne pas se marcher dessus et rester sur la même longueur d'onde. La bonne-entente et la coopération est nécessaire et essentielle pour avancer, et nous avons eu la chance de conserver ces avantages sans qui le projet n'aurait pas été mené à bout.