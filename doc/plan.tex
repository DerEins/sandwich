\section{Vue globale du projet}
\subsection{Énoncé et objectif du sujet}
\begin{itemize}
\item Présentation générale du sujet et de ses objectifs 
\end{itemize}
\subsection{Stratégie globale d'implémentation}
\begin{itemize}
\item Présentation des structures et concepts utilisés pour résoudre le sujet
\item Présentation et analyse de l'algorithme globale de résolution (\texttt{project.c})
\item Explication de la stratégie globale des tests
\end{itemize}
\subsection{Organisation et dépendances du projet}
\begin{itemize}
\item Explication de l'organisation du répertoire de travail (\texttt{src},\texttt{tst}...)
\item Présentation des graphes des dépendances et explication de l'utilisation des versions "ext" des interfaces (voir \autoref{fig:GrapheDepSource} et \autoref{fig:GrapheDepTests})
\item Introduction de fonctions et structures nécessaires à la manipulation des objets (\texttt{utils.c} et \texttt{test\_utils.c})
\end{itemize}

\section{Implémentation du monde et de ses règles}
\subsection{Représentation et fonctionnement d'un monde}
\begin{itemize}
\item Présentation des différents mondes implémentés (aléatoire ou non-aléatoire) en fonction des fonctionnalités ajoutées
\item Compatibilité du projet avec l'exécutable \texttt{sdl} pour affichage du jeu grâce à \texttt{world\_display}
\end{itemize}
\subsection{Implémentation et évolution des règles par motifs}
\begin{itemize}
\item Description et analyse de la structure abstraite des règles par motifs
\item Manipulations de ces règles grâce au concept d'interface
\item Implémentation des règles initiales 
\end{itemize}
\subsection{Refonte des règles : des motifs aux fonctions}
\begin{itemize}
\item Limites de la solution des règles par motifs : complexité temporelle élevée pour certain modèles d'implémentation (jeu de la vie notamment)
\item Notion de pointeurs de fonction
\item Application aux règles du jeu de la vie
\end{itemize}

\section{Gestion des changements entre deux mondes}
\subsection{File d'attente des changements}
\begin{itemize}
\item Représentation d'un changement par une structure de type noeud.
\item Choix d'implémentation de la file sous forme de liste chaînée de ces noeuds : structure et méthodes.
\item Complexité des méthodes de la file, optimisation.
\end{itemize}
\subsection{Gestion des conflits}
\begin{itemize}
\item Marquage d'un conflit sur une case d'un monde
\item Analyse de l'algorithme de résolution des conflits au sein de la file (bon choix en terme de complexité temporelle)
\item Intégration de la gestion des conflits au sein du projet.
\end{itemize}
\section{Conclusion}
Retour sur les apports techniques (compétences acquises) et organisationnels (travail de groupe, planification des tâches...) qu'on apporté le projet

